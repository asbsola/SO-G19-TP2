\documentclass{article}
\title{\textbf{Trabajo Práctico 2} \\ [1ex]
\large Instituto Tecnológico de Buenos Aires - Sistemas Operativos (72.11) \\ [1ex]
\large Grupo 19 }
\date{23 de septiembre de 2024}
\author{
\textbf{Ignacio Searles}\\
isearles@itba.edu.ar\\
64.536
\and
\textbf{Augusto Barthelemy Solá}\\
abarthelemysola@itba.edu.ar\\
64.502
\and
\textbf{Santiago Bassi}\\
sabassi@itba.edu.ar\\
64.643
}

\usepackage{multicol}
\usepackage{graphicx, wrapfig}
\graphicspath{ {imagenes/} }

\usepackage{float}
\usepackage{amsmath}
\usepackage{amsfonts}

\usepackage{hyperref}

\usepackage{caption, threeparttable}
\usepackage{hyperref}

\usepackage[margin=1.3in]{geometry}

\renewcommand{\figurename}{Figura}
\renewcommand{\tablename}{Tabla}
\renewcommand{\abstractname}{Resumen}

\begin{document}
\maketitle

\begin {abstract}

El presente informe trata sobre el desarrollo de un kernel que administra los recursos de hardware de una computadora y que tiene una API para interactuar con el espacio de usuario. En el espacio de usuario se desarrolló un shell que permite ejecutar diferentes módulos que tienen el objetivo de mostrar el funcionamiento del sistema.

\end {abstract}

\section {Memory Manager}

Ambos memory managers, al momento de su inicialización, reciben la cantidad de memoria de la que van a disponer. De esa memoria, consumen una parte para almacenar los datos necesarios para su funcionamiento.

Para los testeos del memory manager, se decidio que cuando se compila fuera del kernel, el mismo corra infinitamente pues se puede terminar la ejecucion matando el proceso desde la shell. En cambio, cuando se compila y ejecutan los tests dentro del kernel, se decidió que el mismo corra una cantidad determinada de veces, pues en el sistema implementado no posee procesos por lo que no puede frenar la ejecucion del test.

\subsection {Bitmap Memory Manager}

A continuation se detallan las instrucciones para compilar y ejecutar los test del bitmap memory manager:

\begin{enumerate}

\item Compilacion y ejecucion de los test dentro del kernel:

\begin{enumerate}
    \item En la raiz del proyecto ejecutar \textbf{make}.
    \item Luego ejecutar \textbf{./run.sh}.
    \item En la shell que se abre, ejecutar \textbf{test\_mm}.
\end{enumerate}

\item Compilacion y ejecucion de los test fuera del kernel:

\begin{enumerate}
    \item En la raiz del proyecto ejecutar \textbf{make bitmaptest}.
    \item Luego, ejecutar \textbf{cd Testing}.
    \item Por ultimo, ejecutar \textbf{./bitmapTest \textless memoryAmount\textgreater}. 

    Siendo memoryAmount la cantidad de memoria que se desea asignar.
\end{enumerate}

\end{enumerate}

\subsection {Buddy Memory Manager}


A continuation se detallan las instrucciones para compilar y ejecutar los test del buddy memory manager:

\begin{enumerate}

\item Compilacion y ejecucion de los test dentro del kernel:

\begin{enumerate}
    \item En la raiz del proyecto ejecutar \textbf{make buddy}.
    \item Luego ejecutar \textbf{./run.sh}.
    \item En la shell que se abre, ejecutar \textbf{test\_mm}.
    
\end{enumerate}

\item Compilacion y ejecucion de los test fuera del kernel:

\begin{enumerate}
    \item En la raiz del proyecto ejecutar \textbf{make buddytest}.
    \item Luego, ejecutar \textbf{cd Testing}.
    \item Por ultimo, ejecutar \textbf{./buddytest <memoryAmount>}.

    Siendo memoryAmount la cantidad de memoria que se desea asignar.
\end{enumerate}

\end{enumerate}

\end{document}
